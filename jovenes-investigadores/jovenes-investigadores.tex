\documentclass[a4paper,10pt]{article}
\usepackage[utf8]{inputenc}
\usepackage[spanish]{babel}
\usepackage[hmargin=2.5cm, vmargin=2.5cm]{geometry}
\usepackage{amsmath}
\usepackage{graphicx}
\usepackage{natbib}
\usepackage{url}
\usepackage[pdftex,colorlinks=true]{hyperref}
\hypersetup{
    allcolors=black,
}
\usepackage{mathptmx} %Times font


\title{
    \textbf{
    TESSEROIDES CON DENSIDADES VARIABLES:
    CÁLCULO DE CAMPOS GRAVITATORIOS EN COORDENADAS ESFÉRICAS CON
    DENSIDADES VARIABLES EN PROFUNDIDAD
    }
}
\author{
    Santiago R. Soler$^{1,2}$ \vspace{0.5em} \\ 
    \textit{$^1$ Instituto Geofísico Sismológico Volponi, Universidad Nacional de San Juan} \\
    \textit{$^2$ Consejo Nacional de Investigaciones Científicas y Técnicas (CONICET)} \vspace{0.4em} \\
    email: santiago.r.soler@gmail.com
}
\date{}


\begin{document}

\maketitle

\vspace{-2.5em}
\begin{center}
\textbf{Palabras Clave:} poner algo
\end{center}
\vspace{0.5em}

\begin{abstract}
El cálculo directo de campos gravitatorios para grandes extensiones requiere tomar en cuenta la curvatura de la Tierra, lo cual suele resolverse a través del uso de prismas esféricos llamados Tesseroides. Además, estructuras como cuencas sedimentarias presentan variaciones de densidad con la profundidad.

\end{abstract}


\section{Introducción}

Las variaciones de la densidad en la corteza con la profundidad han sido estudiadas por casi un siglo: \citet{Athy1930} obtuvo variaciones exponenciales, mientras que otras funciones fueron propuestas en los siguientes años por \citet{Maxant1980, Rao1986, Rao1993, Rao1994}.
También se ha realizado modelado directo o modelos de inversión gravimétricos con densidades variables, principalmente aplicados en cuencas y fosas \citep{Cordell1973, Rao1986, Cowie1990, Rao1993, Rao1994, Zhang2001, Welford2010}.

Estos modelos directos fueron desarrollados para cuerpos bidimensionales o tridimensionales en coordenadas cartesianas, los cuales se ajustan adecuadamente para aplicaciones de escalas locales.
Sin embargo, la disponibilidad de datos gravimétricos satelitales permite llevar a cabo estudios e interpretaciones a escalas regionales, en las cuales es necesario tomar en cuenta la curvatura de la Tierra a la hora de realizar modelados directos. 

These forward gravity models have been developed for two or three dimensional bodies in cartesian coordinates that properly fit local scales applications.
Nevertheless, the latest satellite missions have provided us gravity measurements with global coverage, which allows geologists and geophysics to perform modelling and interpretation on regional scales.
This raises the importance of building forward models that reproduce the gravity anomalies for such scales.

The main issue that should be overcome is taking into account the curvature of the Earth, thus the forward model must be defined in spherical coordinates.
A common way to achieve this is to discretize the Earth in spherical prisms known as tesseroids, which are defined by pairs of latitude, longitude and radius boundaries (see Figure \ref{fig:tesseroid-uieda}).
Calculating the gravity fields generated by an arbitrary tesseroid on any external point involves the resolution of volume integrals that are generally approximated by numerical computations.
The literature offers two approaches: one involves Taylor series expansion \citep{Heck2007, Grombein2013} while the other makes use of Gauss-Legendre Quadrature (GLQ) \citep{Asgharzadeh2007, Uieda2016, Uieda2017}.
The later consists in approximating the integral by a weighted sum of the effect of point masses located at scaled nodes of the Legendre polynomials.




\bibliographystyle{apalike-es}
\bibliography{../manuscript/bibtex/references}

\end{document}
